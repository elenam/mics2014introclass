\documentclass{beamer}
\usepackage{beamerthemeshadow}
\usepackage{color}
%\usepackage[all]{xy}

\newcommand{\allcomments}[1]{{#1}}
\definecolor{ForestGreen}{RGB}{34,139,34}
\newcommand{\emcomment}[1]{\textcolor{ForestGreen}{\comment{Elena: {#1}}}}
\newcommand{\todo}[1]{\textcolor{blue}{\comment{To Do: {#1}}}}
\newcommand{\pscomment}[1]{\textcolor{red}{\comment{Paul: {#1}}}}
\newcommand{\mmcomment}[1]{\textcolor{magenta}{\comment{Max: {#1}}}}

\mode<presentation>
{
  \usetheme{Copenhagen} %%% Change later
 \usecolortheme{beaver}


  \setbeamercovered{transparent}
  % or whatever (possibly just delete it)
}
\setbeamertemplate{footline}[page number]{}


\begin{document}
\title{Steps towards teaching the Clojure programming language in an introductory CS  class}
\author{Elena Machkasova, Paul Schliep, Max Magnuson}
\institute[UMM] % (optional, but mostly needed)
{
 % \inst{1}%
  University of Minnesota, Morris
}
\date[April 25, 2014]  
{Developing a Graphical Library for a Clojure-based Introductory CS Course 2014.}

\begin{frame}
  \titlepage
\end{frame}

\begin{frame}

  \frametitle{Outline}
\tableofcontents

\end{frame}

\section{Introduction to the Project}

\begin{frame}
\frametitle{The Project}
\begin{itemize}
\item
\end{itemize}
\end{frame}

\begin{frame}
\frametitle{Introduction to Clojure}
\begin{itemize}
\item
\end{itemize}
\end{frame}

\begin{frame}
\frametitle{Clojure Syntax}
\begin{itemize}
\item
\end{itemize}
\end{frame}

\section{Goals and Setup for an Introductory Course}

\begin{frame}
\frametitle{UMM's Introductory CS Course}
\begin{itemize}
\item
\end{itemize}
\end{frame}

\begin{frame}
\frametitle{Introduction to Functional Approaches}
\begin{itemize}
\item
\end{itemize}
\end{frame}

\begin{frame}
\frametitle{Introduction to Racket}
\begin{itemize}
\item
\end{itemize}
\end{frame}

\begin{frame}
\frametitle{Limitations of Clojure}
\begin{itemize}
\item
\end{itemize}
\end{frame}

\begin{frame}
\frametitle{Requirements for a Graphical Library}
\begin{itemize}
\item
\end{itemize}
\end{frame}

\section{Developing a Clojure Graphical Library}

\begin{frame}
\frametitle{Overview of Quil}
\begin{itemize}
\item
\end{itemize}
\end{frame}

\begin{frame}
\frametitle{Developing Programs with Quil}
\begin{itemize}
\item
\end{itemize}
\end{frame}

\begin{frame}
\frametitle{Example of a Quil Program}
\begin{itemize}
\item
\end{itemize}
\end{frame}

\begin{frame}
\frametitle{Issues with Quil}
\begin{itemize}
\item
\end{itemize}
\end{frame}

\section{Our Graphical Library}

\begin{frame}
\frametitle{Development of the Graphical Library}
\begin{itemize}
\item
\end{itemize}
\end{frame}


\begin{frame}
\frametitle{How Our Graphical Library Works}
\begin{itemize}
\item
\end{itemize}
\end{frame}


\begin{frame}
\frametitle{Differences From Racket's Graphical Library}
\begin{itemize}
\item
\end{itemize}
\end{frame}

\begin{frame}
\frametitle{Example of Graphical Library}
\begin{itemize}
\item
\end{itemize}
\end{frame}


\section{Conclusions and Future Work}

\begin{frame}
\frametitle{Conclusions}
\begin{itemize}
\item
\end{itemize}
\end{frame}

\begin{frame}
\frametitle{Future Work}
\begin{itemize}
\item
\end{itemize}
\end{frame}

\begin{frame}
\frametitle{Acknowledgments and selected references}
Selected references:
\end{frame}


\end{document}
