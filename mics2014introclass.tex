% This is sigproc-sp.tex -FILE FOR V2.6SP OF ACM_PROC_ARTICLE-SP.CLS
% OCTOBER 2002
%
% It is an example file showing how to use the 'acm_proc_article-sp.cls' V2.6SP
% LaTeX2e document class file for Conference Proceedings submissions.
% ----------------------------------------------------------------------------------------------------------------
% This .tex file (and associated .cls V2.6SP) *DOES NOT* produce:
%       1) The Permission Statement
%       2) The Conference (location) Info information
%       3) The Copyright Line with ACM data
%       4) Page numbering
%
%  However, both the CopyrightYear (default to 2002) and the ACM Copyright Data
% (default to X-XXXXX-XX-X/XX/XX) can still be over-ridden by whatever the author
% inserts into the source .tex file.
% e.g.
% \CopyrightYear{2003} will cause 2003 to appear in the copyright line.
% \crdata{0-12345-67-8/90/12} will cause 0-12345-67-8/90/12 to appear in the copyright line.
%
% ---------------------------------------------------------------------------------------------------------------
% It is an example which *does* use the .bib file (from which the .bbl file
% is produced).
% REMEMBER HOWEVER: After having produced the .bbl file,
% and prior to final submission,
% you need to 'insert'  your .bbl file into your source .tex file so as to provide
% ONE 'self-contained' source file.
%
% Questions regarding SIGS should be sent to
% Adrienne Griscti ---> griscti@acm.org
%
% Questions/suggestions regarding the guidelines, .tex and .cls files, etc. to
% Gerald Murray ---> murray@acm.org 
%
% For tracking purposes - this is V2.6SP - OCTOBER 2002


\documentclass[12pt]{article}

\setlength{\oddsidemargin}{0in}
\setlength{\evensidemargin}{0in}
\setlength{\topmargin}{0in}
\setlength{\headheight}{0in}
\setlength{\headsep}{0in}
\setlength{\textwidth}{6in}
\setlength{\textheight}{9in}
\setlength{\parindent}{0in} 


\usepackage{listings}
\usepackage{color}

\definecolor{dkgreen}{rgb}{0,0.6,0}
\definecolor{gray}{rgb}{0.5,0.5,0.5}
\definecolor{mauve}{rgb}{0.58,0,0.82}

\usepackage{graphicx} %For jpg figure inclusion
\usepackage{times} %For typeface
\usepackage{epsfig}
\usepackage{color} %For Comments
%\usepackage[all]{xy}
\usepackage{float}
%\usepackage{subfigure} 
\usepackage{url}
\usepackage{parskip}

%% Elena's favorite green (thanks, Fernando!)
\definecolor{ForestGreen}{RGB}{34,139,34}
% Uncomment this if you want to show work-in-progress comments
\newcommand{\comment}[1]{{\bf \tt  {#1}}}
% Uncomment this if you don't want to show comments
%\newcommand{\comment}[1]{}
\newcommand{\emcomment}[1]{\textcolor{ForestGreen}{\comment{Elena: {#1}}}}
\newcommand{\todo}[1]{\textcolor{blue}{\comment{To Do: {#1}}}}

\newcommand{\pscomment}[1]{\textcolor{red}{\comment{Paul: {#1}}}}
\newcommand{\mmcomment}[1]{\textcolor{purple}{\comment{Max: {#1}}}}
\begin{document}
\pagestyle{plain}
%
% --- Author Metadata here ---
%\conferenceinfo{WOODSTOCK}{'97 El Paso, Texas USA}
%\setpagenumber{50}
%\CopyrightYear{2002} % Allows default copyright year (2002) to be
%over-ridden - IF NEED BE. 
%\crdata{0-12345-67-8/90/01}  % Allows default copyright data
%(X-XXXXX-XX-X/XX/XX) to be over-ridden. 
% --- End of Author Metadata ---

\title{Developing a Graphical Library for a Clojure-based Introductory CS Course}
%\subtitle{[Extended Abstract \comment{DO WE NEED THIS?}]
%\titlenote{}}
%
% You need the command \numberofauthors to handle the "boxing"
% and alignment of the authors under the title, and to add
% a section for authors number 4 through n.
%
% Up to the first three authors are aligned under the title;
% use the \alignauthor commands below to handle those names
% and affiliations. Add names, affiliations, addresses for
% additional authors as the argument to \additionalauthors;
% these will be set for you without further effort on your
% part as the last section in the body of your article BEFORE
% References or any Appendices.

\author{
Paul Schliep, Max Magnuson, and Elena Machkasova \\
Computer Science Discipline \\
University of Minnesota Morris\\
Morris, MN 56267\\
schli202@umn.edu, magnu401@umn.edu, elenam@umn.edu
}

\date{}

\maketitle
\thispagestyle{empty}

\section*{\centering Abstract}
UMM introductory CS curriculum emphasizes problem solving and general approaches to programming, such as modularity. Languages in the Lisp family, such as Racket, fit well with this emphasis. Currently, the language Racket is being used in an introductory course at UMM and has beneficial aspects for teaching functional approaches for CS design and a robust graphical library. However, Racket is primarily used as a teaching language and is not often used outside of the classroom setting. In contrast, Clojure, a novel language in the Lisp family, is used more often in real world settings and provides better parallel processing and integration with other languages. Therefore, using Clojure in an introductory class would be potentially more beneficial for students. The focus of our project is developing a graphical library in Clojure for introductory-level students. This is a part of a larger ongoing project to integrate Clojure into a college-level introductory CS course.

Since functional languages focus on functions as programming units, provide abstraction, generalization, modularity, and give a better understanding of recursion (one of the key learning goals in an introductory CS course), we feel it is important that the graphical library used by students implements these functional approaches and is consistent with the functional programming in Clojure. In order to accomplish our goal, we want to create a graphical library that is similar to the one in Racket because it teaches introductory students problem solving skills and modularity with functional approaches in an engaging and entertaining setting. We are developing this library using the existing software package Quil, which is written in Clojure. However, Quil’s design is built upon Java Swing, so it focuses on sequences of commands that directly manipulate memory, which goes against the learning objectives of a class that focuses on functional approaches.

We are developing a collection of functions similar to that of the Model-View Control of Racket’s graphical library on top of Quil’s existing library to address the learning objectives of functional programming and hide the imperative approaches currently in Quil. In this system students program using functions that take a state of the system (such as a position of a game character on a board) and return a new state (such as a new position and/or an image representing the character), without direct memory manipulation.

We have designed and developed a collection of functions that abstract over Quil’s direct manipulation of the system’s state so that the state updates and displays using separate functions in a way that is geared toward introductory-level students. In this paper we present the main set of functions and examples of their usage. 
\emcomment{As submitted to MICS; needs to be changed}


\newpage
\setcounter{page}{1}

\section{Introduction}\label{sec:intro}
\todo{Overview of project and goals, Paul's section}

\section{Overview of Clojure}\label{sec:clojure}
\todo{Max's section}

\section{Goals and Setup for an Introductory Course}\label{sec:racket-clojure}
\emcomment{Section~\ref{subsec:plans} assumes that we have already talked about Clojure-Java integration and usefulness in the work filed. This may or may not belong here, but it has to be somewhere.}

\subsection{Overview of Current UMM CS Introductory Course}\label{subsec:course}
\todo{Elena's subsection}
~\cite{htdp}
~\cite{lein} \emcomment{Look into including URLs}

\subsection{Plans for Introducing Clojure}\label{subsec:plans}
%\todo{Paul's section}
Clojure’s integration with Java and usefulness in the work field are just some of the perks that we feel can be beneficial to students in preparing for their future careers in computer science.  Although Clojure provides many important benefits not seen in Racket, it is not yet a language ready for using in an introductory CS course. In order to make it a usable language, many steps must be taken in order to accommodate for Clojure’s high learning curve that prevents it from being usable for beginner-level CS students.

One barrier preventing Clojure from being used in an introductory CS course is its error messages. Often, they tend to be unintuitive and unnecessarily long and can be hard to understand for experienced programmers, much less beginning CS students. Also, there is not an environment that supports easy navigation of the error messages and connecting them to their location of the offending line of code (such as highlighting that line of code). Since problem solving is a core learning objective of the introductory course, it is essential for error messages to be usable for beginning students to enable them to easily troubleshoot issues and focus on key learning concepts. To solve these issues, we catch the error messages and throw more user-friendly messages appropriate for introductory CS students.  A graphical interface for displaying these transformed error messages in a development environment is also essential to help students on problem solving and learning key components to functional programming.  \emcomment{Here you aren't showing a graphical interface, so I would remove its mention.}

Along with the lack of an appropriate user interface for error messages, there is also an absence of a complete development environment for students to easily program in. The current UMM introductory CS course that uses Racket integrates an IDE called DrRacket that incorporates a user-friendly interface for students to develop programs in, something that is necessary for an introductory CS course.  Currently, there is a development environment \emcomment{say "IDE"; you are using "development" twice} that is still in development called Light Table.  Since Light Table was designed for programming in Clojure and was created with usability in mind, it shows promise in usability by which introductory computer science students can program. In order for this program to work, however, it needs proper leiningen implementation for students to create projects in before it can be useful for an introductory course. \emcomment{I would remove a mention of leiningen and just say that we are currently exploring ways of integrating our error message handling with Light Table.}

Another barrier on integrating Clojure for a beginner-level CS course is that many methods \emcomment{functions} do not make sense in the context \emcomment{terms} of their name, or new functionality should be added to ease the transition to functional programming \emcomment{easing a transition to functional programming sounds like we are going from imperative to functional. Easing introduction to programming for new students?}.  For example, in Racket, there are two functions for adding to a collection: add-first and add-last \emcomment{No, there aren't.}. These two functions make sense in the context of their name and work exactly as they should. In Clojure, you need to first know what kind of collection it is (e.g. a list or sequence) and use the appropriate function to properly add an item to the front or the back of the collection or list. This can potentially be uneasy for students to work with, but with add-first and add last functions where they work as intended disregarding what kind of collection it is \emcomment{It wouldn't be clear where this is coming from. Mention that some collections add elements to the front, and some to the end}, which can make transitioning to understanding programming a simpler process.

\emcomment{Some of this stuff probably needs to move since it's not directly related to the imperative/functional stuff}

\begin{verbatim}         
(add-first 1 [2 3 4]) ;;returns a sequence of [1 2 3 4]
(add-last 4 '(1 2 3)) ;; returns a list of '(1 2 3 4)

(cons 1 [2 3 4])
(conj [1 2 3] 4)
(cons 4 '(1 2 3))
(conj '(2 3 4) 1)
\end{verbatim}

\todo{add an example of a function that should be renamed?}

In order to teach many fundamentals of an introductory CS course, it is crucial that students work with and learn about graphical manipulation. Currently, there is no built in system for graphical functions in Clojure and there is no beginner-friendly graphical library that teaches the learning objectives of functional programming in an introductory CS course. A potential graphical library to use for programming is an open-source project called Quil that has many of the functions necessary for students to use in an introductory CS course. However, there are limitations and potential issues to using this graphical library that will steer away from the learning objectives and functional approaches that UMM’s introductory CS curriculum has in place.

\subsection{Requirements for a Graphical Library}\label{subsec:requirements}
\todo{Talk about state and how to handle in a functional way, Max's section}

\section{Developing a Clojure Graphical Library}\label{sec:library}

\subsection{Overview of Quil}\label{subsec:quil}
\todo{technical difficulties, imperative approaches, etc. Paul's section}

\subsection{Introducing Functional Approaches}\label{subsec:functional}
\todo{Max's section}

\section{Examples of Usage of the Graphical Library}\label{sec:usage}
\todo{Add examples as we are writing. Make it into a section once we know what's in it.}

\section{Conclusions and Future Work}\label{sec:future-work}


%
% The following two commands are all you need in the
% initial runs of your .tex file to
% produce the bibliography for the citations in your paper.
%\bibliographystyle{abbrv}
%\end{thebibliography}

%\bibliography{generic_types}  
% You must have a proper ".bib" file
%  and remember to run:
% latex bibtex latex latex
% to resolve all references
%
% ACM needs 'a single self-contained file'!
%
\bibliographystyle{ACM}
\bibliography{mics2014introclass}


% That's all folks!
\end{document}

%%%%%%%%%%%%%%%%%%%%%%%%%%%%%%%%%%%%%%%%%%%%%%%%%%%%%%%%%%%%%%%%
